\documentclass{jarticle}

% プリアンブル
\title{練習用}
\author{小沼悠}
\date{\today}
\usepackage[dvipdfmx]{graphicx}

\begin{document}
  % タイトルを出力
  \maketitle

\section{自己紹介}
\subsection{氏名}
小沼 悠\cite{2}
\subsection{生年月日}
2000/01/24\cite{2}
\section{画像を張る}
% 図の挿入
\begin{figure}[htbp]
\begin{center}
\includegraphics[width=50mm]{pompom11}
\caption{ポムポムプリンとマフィン}
\end{center}
\end{figure}
\cite{2}
\begin{tabular}{l}
\\
\\
\\
\end{tabular}
\section {表を作成する}
%表の挿入
\begin{table}[h]
%\label{table:data_type}
\begin{flushleft}
  \begin{tabular}{|l|c|r|r|}
    メニュー & サイズ & 値段 & カロリー \\
    牛丼 & 並盛 & 500円 & 600 kcal \\
    牛丼 & 大盛 & 1,000円 & 800 kcal \\
    牛丼 & 特盛 & 1,500円 & 1,000 kcal \\
    牛皿 & 並盛 & 300円 & 250 kcal \\
    牛皿 & 大盛 & 700円 & 300 kcal \\
    牛皿 & 特盛 & 1,000円 & 350 kcal\\
  \end{tabular}
\caption{牛丼表}
\cite{1}
\end{flushleft}
\end{table}

\section {参考文献を入れる}
\begin{thebibliography}{9}
\bibitem{1}http://www.latex\_cmd.com/fig\_tab/table01.html(表組みの基本‐LaTeXコマンド集)
\bibitem{2}https://www.comp.tmu.ac.jp/tsakai/lectures/intro-tex.html\#article(Tex入門)
\end{thebibliography}

\end{document}